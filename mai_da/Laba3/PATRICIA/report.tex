\documentclass[12pt]{article}

\usepackage{fullpage}
\usepackage{multicol,multirow}
\usepackage{tabularx}
\usepackage{ulem}
\usepackage[utf8]{inputenc}
\usepackage[russian]{babel}

\begin{document}

\section*{Лабораторная работа №\,3 по курсу дискретного анализа: Исследование качества программ}

Выполнил студент группы М8О-208Б-17 МАИ \textit{Гринин Вячеслав}.

\subsection*{Условие}

Для реализации словаря из предыдущей лабораторной работы, необходимо провести исследование скорости выполнения и потребления оперативной памяти.
В случае выявления ошибок или явных недочётов, требуется их исправить.

Должны быть использованы утилиты: \textbf{gprof}, \textbf{valgrind}, \textbf{dmalloc}.


\subsection*{Метод решения}

В процессе решения лабораторной работы №2, отмечать места возможных утечек памяти и отмечать ускорение работы функций, которые удалось оптимизировать.

\subsection*{Дневник отладки}

По классике - наблюдались проблемы с утечками памяти. Это можно увидеть из сообщений утилиты valgrind и ключей -v --leak-check=full:

{\texttt{
==4682== HEAP SUMMARY:\\
==4682==     in use at exit: 128,304 bytes in 1,002 blocks\\
==4682==   total heap usage: 4,005 allocs, 3,003 frees, 434,032 bytes allocated\\
==4682== \\
==4682== Searching for pointers to 1,002 not-freed blocks\\
==4682== Checked 113,264 bytes\\
==4682== \\
==4682== 304 (48 direct, 256 indirect) bytes in 1 blocks are definitely lost in loss record 2 of 3\\
==4682==    at 0x4838E86: operator new(unsigned long) (vg_replace_malloc.c:344)\\
==4682==    by 0x4028CE: TPatriciaTrie<unsigned long long>::TPatriciaTrie() (in /home/mimka/GitHub/maiLabs/mai_da/Laba3/PATRICIA/main/main)\\
==4682==    by 0x404CB6: main (in /home/mimka/GitHub/maiLabs/mai_da/Laba3/PATRICIA/main/main)\\
==4682== 
==4682== 128,000 bytes in 1,000 blocks are definitely lost in loss record 3 of 3\\
==4682==    at 0x483880B: malloc (vg_replace_malloc.c:309)\\
==4682==    by 0x4C2166D: strdup (in /usr/lib64/libc-2.28.so)\\
==4682==    by 0x402513: TPatriciaTrieItem<unsigned long long>::Initialize(char*, unsigned long long, int, TPatriciaTrieItem<unsigned long long>*, TPatriciaTrieItem<unsigned long long>*) (in /home/mimka/GitHub/maiLabs/mai_da/Laba3/PATRICIA/main/main)\\
==4682==    by 0x402CA5: TPatriciaTrie<unsigned long long>::Insert(char*, unsigned long long) (in /home/mimka/GitHub/maiLabs/mai_da/Laba3/PATRICIA/main/main)\\
==4682==    by 0x40405A: Benchmark(TPatriciaTrie<unsigned long long>*, unsigned long long) (in /home/mimka/GitHub/maiLabs/mai_da/Laba3/PATRICIA/main/main)\\
==4682==    by 0x404CC7: main (in /home/mimka/GitHub/maiLabs/mai_da/Laba3/PATRICIA/main/main)\\
==4682== \\
==4682== LEAK SUMMARY:\\
==4682==    definitely lost: 128,048 bytes in 1,001 blocks\\
==4682==    indirectly lost: 256 bytes in 1 blocks\\
==4682==      possibly lost: 0 bytes in 0 blocks\\
==4682==    still reachable: 0 bytes in 0 blocks\\
==4682==         suppressed: 0 bytes in 0 blocks\\
==4682== \\
==4682== ERROR SUMMARY: 2 errors from 2 contexts (suppressed: 0 from 0)\\
}}

Исходя из отчёта valgrind можно заметить, что мы выделяли память под строки, но не освободили её. Так же некорректно работал деструктор класса TPatriciaTrie. Память, выделенная под дерево, не освобождалась полностью. После исправлений мы заглянем в отчёт dmalloc, который выполняет схожую функцию - обнаружение утечек памяти.

{\texttt{
		1554628449: 120012: Dmalloc version '5.5.2' from 'http://dmalloc.com/'\\
		1554628449: 120012: flags = 0x4e48503, logfile 'logfile'\\
		1554628449: 120012: interval = 100, addr = 0, seen # = 0, limit = 0\\
		1554628449: 120012: starting time = 1554628439\\
		1554628449: 120012: process pid = 3811\\
		1554628449: 120012: Dumping Chunk Statistics:\\
		1554628449: 120012: basic-block 4096 bytes, alignment 8 bytes\\
		1554628449: 120012: heap address range: 0x7f4d67474000 to 0x7f4d688af000, 21213184 bytes\\
		1554628449: 120012:     user blocks: 2877 blocks, 11782144 bytes (74\%)\\
		1554628449: 120012:    admin blocks: 978 blocks, 4005888 bytes (25\%)\\
		1554628449: 120012:    total blocks: 3855 blocks, 15790080 bytes\\
		1554628449: 120012: heap checked 1201\\
		1554628449: 120012: alloc calls: malloc 60006, calloc 1, realloc 0, free 60005\\
		1554628449: 120012: alloc calls: recalloc 0, memalign 0, valloc 0\\
		1554628449: 120012: alloc calls: new 0, delete 0\\
		1554628449: 120012:   current memory in use: 73728 bytes (2 pnts)\\
		1554628449: 120012:  total memory allocated: 6322776 bytes (60007 pnts)\\
		1554628449: 120012:  max in use at one time: 6322776 bytes (60007 pnts)\\
		1554628449: 120012: max alloced with 1 call: 160000 bytes\\
		1554628449: 120012: max unused memory space: 5450728 bytes (46\%)\\
		1554628449: 120012: top 10 allocations:\\
		1554628449: 120012:  total-size  count in-use-size  count  source\\
		1554628449: 120012:     2560000  20000           0      0  TPatriciaTrieItem.cpp:20\\
		1554628449: 120012:     2560000  20000           0      0  main.cpp:31\\
		1554628449: 120012:      160000      1           0      0  main.cpp:22\\
		1554628449: 120012:         256      1           0      0  TPatriciaTrie.cpp:15\\
		1554628449: 120012:      773448  15913      773448  15913  Other pointers\\
		1554628449: 120012:     6053704  55915      773448  15913  Total of 4\\
		1554628449: 120012: Dumping Not-Freed Pointers Changed Since Start:\\
		1554628449: 120012:  not freed: '0x7f4d67478808|s1' (1024 bytes) from 'unknown'\\
		1554628449: 120012:  not freed: '0x7f4d68873008|s1' (72704 bytes) from 'unknown'\\
		1554628449: 120012:  total-size  count  source\\
		1554628449: 120012:           0      0  Total of 0\\
}}

В отчёте можно заметить, что у нас имеется не освобождённая память, но не понятно, из какой функции её забрали. Valgrind же свою очередь не видит утечек памяти: 

{\texttt{
		[mimka@localhost main]\$ valgrind --tool=memcheck ./main\\
		==3858== Memcheck, a memory error detector\\
		==3858== Copyright (C) 2002-2017, and GNU GPL'd, by Julian Seward et al.\\
		==3858== Using Valgrind-3.14.0 and LibVEX; rerun with -h for copyright info\\
		==3858== Command: ./main\\
		==3858== \\
		==3858== HEAP SUMMARY:\\
		==3858==     in use at exit: 0 bytes in 0 blocks\\
		==3858==   total heap usage: 20,005 allocs, 20,005 frees, 1,042,520 bytes allocated\\
		==3858== \\
		==3858== All heap blocks were freed -- no leaks are possible\\
		==3858== \\
		==3858== For counts of detected and suppressed errors, rerun with: -v\\
		==3858== ERROR SUMMARY: 0 errors from 0 contexts (suppressed: 0 from 0)\\
}}
\\
Встаёт вопрос кому из них верить. И если утечка действительно есть, то не совсем понятно как от неё избавиться. Я решил оставить это и перейти к оптимизации. Программа не проходила чекер в силу своей медлительности. Имелись подозрения на счёт сохранения и загрузки дерева в файл, и я написал альтернативный способ сохранения и загрузки, оставив и старый способ. Прогнав программу по бенчмарку, запустил gprof. Результаты подтвердили мои предположения:
\\


\subsection*{Выводы}

dmalloc - программная библиотека, которая позволяет уловить некорректную работу с памятью. Её работа строится на подмене библиотечных функций работы с памятью своими. Под некорректной работой с памятью подразумеваются - утечки памяти, выход за пределы массива, который был выделен где-то в куче (ошибка off-by-one). Но в случае с массивом, расположенном в стеке - она не сможет определить ошибку off-by-one.\\
valgrind - утилита, которая обеспечивает те же функции, что и dmalloc. Отличается способом работы. Если в случае dmalloc мы подключали отдельную библиотеку, перекомпилировали код с нужными ключами, запускали его и смотрели логи, то в случае с valgrind - просто запускаем готовый код без лишних махинаций, а valgrind в свою очередь выполняет некоторые преобразования в нём, чтобы в зависимости от активированных инструментов выполнить задачу. Так же valgrind имеет ту же проблему, что и dmalloc - он не в состоянии увидеть выход за пределы массива, который находится в стеке. Хотя, если верить Википедии, то существует экспериментальная утилита, которая позволяет это увидеть.\\
Обе утилиты увеличивают время работы кода, так что уж точно не стоит делать замеры производительности в них.\\
gprof - утилита, специализирующаяся на вычислении времени работы функций в коде. Для работы с ней надо применить соответствующий ключ во время компиляции, потом просто запустить код. Подождать, пока он завершит свою работу. Затем уже применяем gprof. Это необходимо, так как во время работы исполняемого кода, скомпилированного должным образом, составляется граф вызовов, который уже позже читается утилитой gprof и интерпретируется в нужном формате.\\
По итогу могу сказать такие вещи - с gprof я пока плохо разобрался, так как мне не до конца ясны некоторые моменты, начиная от того, как он вычисляет время работы функций, и кончая тем, как правильно читать граф вызовов функций. dmalloc мне показался слишком проблемным в плане работы с ним. Тот же valgrind делает всё то же самое, но работа с ним в разы проще. Все эти утилиты крайне полезны для программиста и позволяют бороться с утечками памяти, а также совершенствовать и оптимизировать код в целом.

\end{document}

